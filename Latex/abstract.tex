The thesis develops and presents a novel multi-system model that couples a behavioral and an epidemic frameworks. A SIR-like model is combined with a behavioral one, that subdivides the modeled population into three key compartments: Heedless, Compliant, and Against. Instead of being based solely on theoretical principles, the model's features are developed considering real empirical datasets regarding both disease and behavior evolution during the recent COVID-19 pandemic. This approach overcomes the limitations of previous works that implement similar models but use proxies for individual behaviors or rely on the assumption of a direct correlation between opinion and behavior.

Key contributions of the work include:
\begin{itemize}
	\item \textbf{Behavior-epidemic coupling:} The developed model considers peer pressure, fatigue, NPIs' protection, self-isolation, and mean-field parameters representing the activation of government policies to mitigate the disease's spread. These elements contribute to modeling the behavioral transitions of the population and realizing the coupling with the disease model.
	\item \textbf{Analysis:} An extensive analysis of its singular components is conducted, first focusing on the new behavioral model alone and then extending to the full model. This has allowed the development of a foundational framework used with the multi-system full model.
	\item \textbf{Epidemic-reproduction number and simulations:} The epi-behavioral model's reproduction number is calculated using the Next-Generation Matrix method. A full set of simulations to test how the system reacts to different conditions and parameter sets is also performed.
\end{itemize}
The flexibility of the developed model represents a foundation for exploring future scenarios in which social dynamics, like wearing face masks or self-isolating, significantly alter disease outcomes.