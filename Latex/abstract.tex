The thesis develops and presents a novel multi-system model (i.e., a model that simultaneously captures different coexisting and coupled dynamic phenomena) that couples behavioral and epidemic phenomena by combining a SIR-like epidemiological model with a behavioral model. This behavioral model partitions the population into three compartments: \begin{itemize}
	\item[$H$:] Heedless, people careless of the risk associated with the infection;
	\item[$C$:] Compliant, people that want to avoid to becoming infected or infecting others
	\item[$A$:] Against, people who not see the epidemic as a risk and do not use protections or change their behavior during the epidemic. 
\end{itemize}

 The model's features are not only based on theoretical principal, but also developed considering real empirical datasets regarding both disease and behavior evolution during the recent COVID-19 pandemic. This approach overcomes the limitations of previous works that implement similar models, but use proxies for individual behaviors or rely on the assumption of a direct correlation between opinion and behavior.

The key contributions of the work include:
\begin{itemize}
	\item \textbf{Coupling between behaviors and contagion:} The developed model considers phenomena such as peer pressure, fatigue, non-pharmaceutical interventions (NPIs), self-isolation, and mean-field parameters representing government policies to mitigate the spread of the disease. These elements contribute to modeling the behavioral transitions of the population and realizing the coupling with the disease model.
	\item \textbf{Analysis:} An extensive analysis of the model components is conducted, focusing first on the new behavioral model alone and then on the full model. This allows the development of a foundational framework used for the full multi-system model.
	\item \textbf{Epidemic-reproduction number and simulations:} The epi-behavioral model's reproduction number is computed using the Next-Generation Matrix method. A full set of simulations to test the system behavior for different conditions and parameter sets is also performed.
\end{itemize}
The flexibility of the developed model and its ability to capture the interplay between individual behaviors and disease spread offer a foundational tool to explore future scenarios in which social dynamics, such as wearing face masks or self-isolating, significantly alter the epidemic evolution.