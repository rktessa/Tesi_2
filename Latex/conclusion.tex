\chapter{Conclusions}

The work conducted during this thesis has led to the development of a novel epi-behavioral model that couples an SIRS model with a behavioral mean-field model. Both opinion and behavioral modeling have been studied, and the most significant contributions in this field from recent years have been presented to provide a comprehensive perspective on how social dynamics can be integrated into disease progression models.

The main contributions of this study are summarized as follows:
\begin{itemize}
	
\item \textbf{Behavioral-Epidemic Interactions:} The work highlights the complex dynamics that emerge when a disease interacts with human behavior. A behavioral model synthesizing three possible behavioral states during the onset and progression of an epidemic is implemented. From its coupling with the disease model, the role of social dynamics in shaping epidemic outcomes becomes apparent.

\item \textbf{Insights from Simulations:} Through analyses of individual model components and subsequent simulations, key parameters such as the Epidemic Reproduction Number ($E_0$) were identified. The evolution of $E_0$ and its effectiveness as an indicator to predict whether an epidemic will evolve were explored. Furthermore, the model demonstrates various dynamics: while compliance diffusion can mitigate the epidemic's impact, the spread of misinformation, which alters individual behavior, can lead to more severe outcomes. The interplay of behavioral parameters, in particular, creates intriguing influences on epidemic evolution, which are explained in this work.

\item \textbf{Framework Flexibility:} Developed with empirical data in mind, the resulting model is a versatile tool that can be adapted to different scenarios and reflects a wide range of behavioral and epidemiological dynamics.
\end{itemize}