\part{Opinion Disease Model}

\chapter{Analysis of epidemiological and opinion models}
The development of a epidemiological model, that can capture the evolution of a disease influenced by the behaviour of individuals, begins from a study and review of the most significative works already present in this research topic.
These are the different main type of model that have been investigated:
\begin{itemize}
	\item deterministic/mean field models
	\item opinion models
	\item multilayer networks
	\item opinion-disease models	
\end{itemize}

Now it is presented for each of them, the main aspect and knowledge, useful for the development of my model.   

\section{Opinion models}
In the analysis performed by Wang \cite{Wang_2019}, are presented mechanism implemented to explain coevolution spreading in complex network. The principal methodologies created over time are threshold model, that can use a linear threshold or a “Watts threshold”. Here each node has a random different threshold, based on a certain distribution. Using a threshold means that a node change opinion on the basis of its neighbours’ belief. The shape of the network is then fundamental for an opinion to spread. The best scenario is the one in which there is a low degree of randomness, and the network is regular. Also, cluster can have a reinforcement effect, if they are sufficiently connected to the resto of the graph. Their work then report analysis based on competition or cooperation of opinions “contagions”. And a SAR model is presented. Similar to a SIR, here the meaning of letter A is “adopted”. It means becoming convinced of a certain opinion, but with a probability or rate to then return to the previous behaviour. Also the work of \cite{Nunner2021} define and test some different models based on trade-off between the benefit of having connections and the penalty for acquiring infections. It is showed that when the behaviour of people depends on maximizing their net benefit, the individual risk perception plays an important role in the formulation of a cost function. The models derived with this so called co-evolutionary approach, have an overall dynamic very correlated between the two strati: it is a feedback loop between infection spreading, people behaviour adaptation and consequently structural modification in the network.


\section{Multilayer network}
One work based on feedback between two networks concatenated is the one performed by Peng et all, \cite{Peng2021}. Here there is explained a model based on two graphs, where one simulates the evolution of a disease, using a SIR or SIRS dynamic, and another explicit the behaviour of individual in a UPAU network. U means uninformed, P pro-physical distancing and A anti-physical distancing.  In this network the people’s conduct influence the $\beta$ coefficient of the epidemic diffusion. They demonstrate the effectiveness of having an opinion in reducing the negative effect of a disease and that lengthening the duration time for which an individual maintains opinion can help suppressing the transmission.
Study the effect of competition in a multilayer network is the objective of Teslya et all research \cite{teslya2022}. At cause of interpersonal communication individual can change their opinion. They are divided in two main groups, positive or negative w.r.t health conduct. Here is also inserted an influence due to assortatively when contacting with others. Their principal results further than the fact that opinion influence disease, is realizing a model in which the two opinions can coexist at equilibrium. There can be oscillation of prevalence due to increased transmissibility of infection. In SIR model they demonstrate a reverse correlation between the rate of social contact and the peak magnitude of infectious. The causes of oscillations in the disease dynamic are a high infection rate and a pronounced difference in infection rate between individuals with different opinions. The others important factors are the high-rate opinion exchange and high sensitivity of population to prevalence. 
In the article \cite{ Alvarez_Zuzek_2017} the opinion about vaccination is taking in consideration, into a SIR+V mean field model. Conversating is the mean used by individual to modify their opinion. With a very positive opinion susceptible individuals can choose to take the vaccine. Interesting they use a r factor to describe the extremism in opinion. Varying this coefficient, they observe that the best scenario for delay the development of an epidemic is the one where the society is neutral. So, when there aren’t compromise or persuasion, but the conversation is based on “rational” arguments.  Another works analysing two competing opinion is \cite{Epstein_2021}. Here population is sensitive to both fear of vaccine and disease. These two interact and the vaccination grow rate increases only if the fear of the disease is larger than the of vaccine.  The infection curve is very influenced by the presented dynamic, and the best scenario is obviously the one in which the fear of vaccine does not exist. However, in the case where the two fears coexist there is an improvement in the number of infected, for multiple infection waves.
The work by Auld \cite{Auld_2003}, reflect an observed characteristic in society: pessimistic expectations over the future induce a more risky behaviours. This conclusion derives observing and simulation evolution correlated to the news about a vaccine. This knowledge causes a decrease in infection rate before the vaccine becomes available. Then there is a return to normal behaviour. If there are not information, pessimism cause more risky behaviour. 
In \cite{Sontag_2022} there is another SIR and opinion model, with population that is divided in trusting and distrusting. They add in the model the effect of fading and a global force, that simulates central interventions. The main interesting conclusion of their work is that strong public intervention have a similar effect to the network to the ones obtained if the population is composed of trusting and compliant individuals. However, higher percentages of distrusting cause the model to pass a phase transition where outbreaks cannot be suppressed. 
A different approach in using a multilayer network is the one realised in \cite{Turker_2023}, where the social structure of a town is re-created. Every layer describes the places populated by individuals: from house, to work, distinguishing between different type of work, and considering a level for friendship. Each person is present to more than a layer and, in each layer, relates to different agents, based on the social group’s provenience.  Using this approach, they have found that the level in which is easier for an outbreak to develop is the one associated with friendship. Here the interaction is closer with others, the security level is lower. For this reason, a lower value of transmissibility rate $\beta$ is sufficient to have an epidemic with many susceptible involved. 

\section{Opinion-disease model}
The work done by Funk and its colleagues \cite{ Funk_2010}, it is very interesting: they collect and explain systematically the behavioural reaction of population in response to a pandemic. They classify the human behaviour subject to different possible sources of information. An information can be global available or local. This reflects the way it radiates or if develops in social cluster. Another important difference is related to objectivity. Certain information is based on belief and can change with time. This typology can be influenced by the social connections of an individual or by the influence of external agents, like media. Cognitive bias also can have an impact on our opinions: amplification, confirmation, anchoring bias. They then focus on the influence of self-initiated action in the control of disease diffusion. When an individual change its behaviour, form a modelling point of view this can influence: its probability to change state (from S to I for example). The value of $\beta$ or $\gamma$. Modification in the contact network, with a self-isolation or adherence to more cautious conduct. Fear has an important role in how people face epidemic. Due to this emotion, people can decide to get vaccinated for example (or not, if they are more frightened by vaccines). Another phenomenon observed and influenced by fear is saturation. When there is many infectious people tend to decrease their number of contact and this cause a decrement in the I curve.  Another multilayer network with two opinion, 0 where individual not take precautions and 1 where the protective measures are used, is presented in \cite{Frieswijk_2022}. This model is associated to a SIS disease one. The article studies the stability of asymptotically equilibria of the system. Assuming different value of a parameter used to describe risk perception they found a set of final possible states. The most interesting is a stable asymptotical equilibrium in which there is a periodic epidemic outbreak and a consequently population behaviour response, changing behaviour to a safer.
An analysis of people choices about vaccinations is done by \cite{Bauch_2012}, they study the feedback between the positive effect due to vaccination and the fear of being vaccinated. In fact, thanks to vaccines, the disease incidence can become very low, and the perception of risk related to them can seem larger. They implemented an approach based on game theory and using social learning.
A possibility to integrate the effect of opinion in the dynamic of an epidemic, is creating different subgroups of susceptible. They are separated according to their level of opinion, and the less they belief in use of NPI, for example, the higher probability of being infectious they have. This is the approach used in \cite{Tyson_2020}. They also implemented different functions describing the influence between opinion and the possibility to become infected. 
The influence of media has also been analysed. This is interesting, because it’s a communication channel that can be used by government, and so it is an available control measure that can be implemented, to try control the behaviour of population.  For example in \cite{Collinson2014} a parameter depending on I value simulates the effect of media covering the news about the disease. Increasing the number of infectious cause, the creation of news and other media about it. These can have as effect to induce more people practice social distance for example. Study both the effect of media, see as a central node of communication joined with opinion evolution is done in \cite{Granell_2014}. Nodes co-exist into two layer, one for disease spreading and one for awareness, (unaware-aware-unaware model). In their application the awareness process without media, must reach a certain level on the transmissibility of awareness to influence the onset of epidemic. Instead, with an influence of media, greater than zero, this “metacritical” point disappears. A central broadcast, even with a small communication influence power, as a direct effect on all the network dynamic. 



